% Options for packages loaded elsewhere
\PassOptionsToPackage{unicode}{hyperref}
\PassOptionsToPackage{hyphens}{url}
\documentclass[12pt, ]{article}

\usepackage{mathtools}
\usepackage{amsmath}
\usepackage{amsthm}
\usepackage{amssymb}
\usepackage[italicdiff]{physics}
\mathtoolsset{showonlyrefs}

% SPACING AND FONTS %%%%%%%%%%%%%%%%%%%%%%%%%%%%%%%%%%%%%%%%%%%%%%%%%%%%%%%%%%%%
\usepackage{iftex}
% CAREFUL: the order of font includes here is very important!
\ifPDFTeX
  \usepackage[OT1,T1]{fontenc}
  \usepackage[utf8]{inputenc}
  \usepackage{textcomp} % provide euro and other symbols
    \usepackage[p,osf,swashQ]{cochineal}
  \usepackage[cochineal,vvarbb]{newtxmath}
      \usepackage[scale=0.95]{biolinum}
    \usepackage[scale=0.95,varl]{inconsolata}
\else % if luatex or xetex
  \usepackage[scale=0.95,varl]{inconsolata}
  \usepackage{newpxtext}
  \usepackage{mathpazo}
    \usepackage[scale=0.95]{biolinum}
  \fi
\ifLuaTeX
  \usepackage{selnolig}  % disable illegal ligatures
\fi
\IfFileExists{microtype.sty}{% use microtype if available
  \usepackage[]{microtype}
  \UseMicrotypeSet[protrusion]{basicmath} % disable protrusion for tt fonts
}{}

\setlength{\parindent}{0pt}
\setlength{\parskip}{10pt plus 2pt minus 2pt}
\setlength{\emergencystretch}{3em} % prevent overfull lines
\widowpenalty=10000
\clubpenalty=10000
\flushbottom
\allowdisplaybreaks
\sloppy


% CORE PACKAGES %%%%%%%%%%%%%%%%%%%%%%%%%%%%%%%%%%%%%%%%%%%%%%%%%%%%%%%%%%%%
\usepackage[dvipsnames,svgnames,x11names]{xcolor}
\usepackage[lmargin=1.5in,rmargin=1.5in,tmargin=1.2in,bmargin=1.2in]{geometry}
\usepackage[format=plain,
  labelfont={bf,sf,small,singlespacing},
  textfont={sf,small,singlespacing},
  justification=justified,
  margin=0.25in]{caption}

% SECTIONS AND HEADINGS %%%%%%%%%%%%%%%%%%%%%%%%%%%%%%%%%%%%%%%%%%%%%%%%%%%%%%%%
\setcounter{secnumdepth}{4}
\usepackage{sectsty}
\usepackage[compact]{titlesec}
% short title
\makeatletter
\newcommand\@shorttitle{}
\newcommand\shorttitle[1]{\renewcommand\@shorttitle{#1}}
\usepackage{fancyhdr}
\fancyhf{}
\pagestyle{fancy}
\renewcommand{\headrulewidth}{0pt}
\fancyheadoffset{0pt}
%\lhead{\scshape \@shorttitle}
%\rhead{\scshape\today}
\cfoot{\thepage}
\makeatother
% abstract styling
\renewenvironment{abstract}{
  \centerline
  {\large\sffamily\bfseries Abstract}\vspace{-1em}
  \begin{quote}\small
}{
  \end{quote}
}

% PANDOC INCLUDES %%%%%%%%%%%%%%%%%%%%%%%%%%%%%%%%%%%%%%%%%%%%%%%%%%%%%%%%%%%%%%

\providecommand{\tightlist}{%
  \setlength{\itemsep}{0pt}\setlength{\parskip}{0pt}}\usepackage{longtable,booktabs,array}
\usepackage{calc} % for calculating minipage widths
% Correct order of tables after \paragraph or \subparagraph
\usepackage{etoolbox}
\makeatletter
\patchcmd\longtable{\par}{\if@noskipsec\mbox{}\fi\par}{}{}
\makeatother
% Allow footnotes in longtable head/foot
\IfFileExists{footnotehyper.sty}{\usepackage{footnotehyper}}{\usepackage{footnote}}
\makesavenoteenv{longtable}
\usepackage{graphicx}
\makeatletter
\def\maxwidth{\ifdim\Gin@nat@width>\linewidth\linewidth\else\Gin@nat@width\fi}
\def\maxheight{\ifdim\Gin@nat@height>\textheight\textheight\else\Gin@nat@height\fi}
\makeatother
% Scale images if necessary, so that they will not overflow the page
% margins by default, and it is still possible to overwrite the defaults
% using explicit options in \includegraphics[width, height, ...]{}
\setkeys{Gin}{width=\maxwidth,height=\maxheight,keepaspectratio}
% Set default figure placement to htbp
\makeatletter
\def\fps@figure{htbp}
\makeatother
% END PANDOC %%%%%%%%%%%%%%%%%%%%%%%%%%%%%%%%%%%%%%%%%%%%%%%%%%%%%%%%%%%%%%%%%%%

% USER INCLUDES %%%%%%%%%%%%%%%%%%%%%%%%%%%%%%%%%%%%%%%%%%%%%%%%%%%%%%%%%%%%%%%%
% additional LaTeX code for the "preamble" goes here
\makeatletter
\makeatother
\makeatletter
\makeatother
\makeatletter
\@ifpackageloaded{caption}{}{\usepackage{caption}}
\AtBeginDocument{%
\ifdefined\contentsname
  \renewcommand*\contentsname{Table of contents}
\else
  \newcommand\contentsname{Table of contents}
\fi
\ifdefined\listfigurename
  \renewcommand*\listfigurename{List of Figures}
\else
  \newcommand\listfigurename{List of Figures}
\fi
\ifdefined\listtablename
  \renewcommand*\listtablename{List of Tables}
\else
  \newcommand\listtablename{List of Tables}
\fi
\ifdefined\figurename
  \renewcommand*\figurename{Figure}
\else
  \newcommand\figurename{Figure}
\fi
\ifdefined\tablename
  \renewcommand*\tablename{Table}
\else
  \newcommand\tablename{Table}
\fi
}
\@ifpackageloaded{float}{}{\usepackage{float}}
\floatstyle{ruled}
\@ifundefined{c@chapter}{\newfloat{codelisting}{h}{lop}}{\newfloat{codelisting}{h}{lop}[chapter]}
\floatname{codelisting}{Listing}
\newcommand*\listoflistings{\listof{codelisting}{List of Listings}}
\makeatother
\makeatletter
\@ifpackageloaded{caption}{}{\usepackage{caption}}
\@ifpackageloaded{subcaption}{}{\usepackage{subcaption}}
\makeatother
\makeatletter
\@ifpackageloaded{tcolorbox}{}{\usepackage[skins,breakable]{tcolorbox}}
\makeatother
\makeatletter
\@ifundefined{shadecolor}{\definecolor{shadecolor}{rgb}{.97, .97, .97}}
\makeatother
\makeatletter
\makeatother
\makeatletter
\makeatother
% END USER INCLUDES %%%%%%%%%%%%%%%%%%%%%%%%%%%%%%%%%%%%%%%%%%%%%%%%%%%%%%%%%%%%

% BIBLIOGRAPHY %%%%%%%%%%%%%%%%%%%%%%%%%%%%%%%%%%%%%%%%%%%%%%%%%%%%%%%%%%%%%%%%%
\usepackage[]{natbib}
\bibliographystyle{apalike}

% Give it this name so that it works with ::: #refs
\newenvironment{CSLReferences}[2]{
\bibliography{bibliography.bib}
\clearpage
}{}

% LINKS %%%%%%%%%%%%%%%%%%%%%%%%%%%%%%%%%%%%%%%%%%%%%%%%%%%%%%%%%%%%%%%%%%%%%%%%
\usepackage{hyperref}
\usepackage{url}
\hypersetup{
  pdftitle={Using Difference in Differences to Understand the Effect of the Affordable Care Act (ACA) on the demographics of Federally Qualified Health Centers (FQHCs)},
  pdfauthor={Aparajita Kaphle},
  colorlinks=true,
  linkcolor={black},
  filecolor={Maroon},
  citecolor={VioletRed4},
  urlcolor={DodgerBlue4},
  pdfcreator={LaTeX via pandoc}}

% TITLE, AUTHOR, DATE %%%%%%%%%%%%%%%%%%%%%%%%%%%%%%%%%%%%%%%%%%%%%%%%%%%%%%%%%%
\title{\sffamily\bfseries\huge\parfillskip=0pt
\rightskip=0pt plus .5\textwidth
\leftskip=0pt plus .5\textwidth
\emergencystretch=.3\textwidth Using Difference in Differences to
Understand the Effect of the Affordable Care Act (ACA) on the
demographics of Federally Qualified Health Centers (FQHCs)}
\shorttitle{Using Difference in Differences to Understand the Effect of the Affordable Care Act (ACA) on the demographics of Federally Qualified Health Centers (FQHCs)}
\author{\textbf{Aparajita Kaphle}
 }
\date{}


\begin{document}
\allsectionsfont{\sffamily}

\maketitle

\begin{abstract}
Recent studies have argued that the 340B program and subsequent
expansions that increase eligibility has actually lead to health centers
(FQHCs) opening up in wealthy areas rather than the low-income areas
they were designed to help. Using the Affordable Care Act as a treatment
variable, I utilize a difference in differences test to understand if
adoption of the ACA and its expansions of FQHC eligbility has lead to
more FQHCs opening in wealthier areas. I find that in states where the
ACA was implemented, it has lowered the income of where FQHCs are
located compared to states where ACA was not implemented.This indicates
that perhaps FQHC eligibility expansion will allow the 340B program to
work as intended.
\end{abstract}

\ifdefined\Shaded\renewenvironment{Shaded}{\begin{tcolorbox}[borderline west={3pt}{0pt}{shadecolor}, interior hidden, boxrule=0pt, enhanced, sharp corners, breakable, frame hidden]}{\end{tcolorbox}}\fi



% USER BODY %%%%%%%%%%%%%%%%%%%%%%%%%%%%%%%%%%%%%%%%%%%%%%%%%%%%%%%%%%%%%%%%%%%%

\hypertarget{introduction}{%
\section{Introduction}\label{introduction}}

(See github link: \url{https://github.com/akaphle01/AQRDFINALPROJ})

In this investigation I plan to employ a difference in differences model
to analyze the effects of the Affordable Care Act (ACA) on the
demographics of where FQHCs are opening up. I will be comparing the
outcomes between states that adopted the ACA vs States that did not
adopt the ACA using 2011 and 2019 as my before and after treatments. The
ACA which was passed in 2010 but implemented in 2014 was adopted by some
states and not by others giving me the experimental setting to examine
if FQHCs opened in 2019 in ACA adoption states were located in wealthier
areas than their non-ACA adoption states. To do so I will be treating
ACA implementation as the treatment, the treatment group to be the
States that implemented ACA and the control group as the non-ACA group.
It should be acknowledged that the states that did not implement the ACA
were largely Southern states and to try and control for this regional
variation I have also included controls that take into account the
rural/urban demography of an FQHC location as well as other variables
such as unemployment. I also implemented a fixed effects model but
lacking a lengthy enough panel data, I ultimately decided upon relying
on One-way Linear Regression that later I carried out validation tests.

\hypertarget{literature-review}{%
\section{Literature Review}\label{literature-review}}

Enacted in 1992 to support rural hospitals and health centers that serve
disadvantaged communities, the 340B program has come under fire in
recent years as many hospitals and health centers have abused the
program for their profit.\footnote{``A Little-Known Windfall for Some
  Hospitals, Now Facing Big Cuts'' 2019. \emph{nytimes.com}.
  \url{https://www.nytimes.com/2018/08/29/upshot/a-little-known-windfall-for-some-hospitals-now-facing-big-cuts.html}.}
The passage of the Affordable Care Act allowed for an expansion of what
health centers qualify as an FQHC and as a result, the number of FQHCs
has rocketed. With this increase, some journalistic and empirical
research indicates that FQHCs registered in the 340B program have
steadily risen in wealthy and affluent neighborhoods \citep{conti2014}.

The 340B program requires pharmaceutical companies to provide discounts
for healthcare facilities offering care to low-income and disadvantaged
communities. These discounts are most useful when treating uninsured
residents as many patients are not able to pay out of pocket for some
medication. However, if healthcare facilities within the 340B program
treat someone with insurance, the hospital is able to charge insurance
for the full cost of the drug and reap the profits between the
discounted drugs and the insurance payment. This profit was originally
intended to assist healthcare facilities in their day-to-day upkeep.

However, over the years, lack of reporting on the investments of
healthcare facilities and expansions of eligibility into the 340B
program has allowed healthcare facilities to open clinics in wealthy
areas, serving primarily affluent communities and pocketing the profits
generated from a heavily-insured population. Since its implementation,
the program has only expanded, allowing Family Planning Centers to be
eligible in 1998, children's hospitals in 2006. By 2017 there were over
12,000 covered entities and as of 2015, 40\% of all US hospitals were
enrolled as 340B entities \citep{mulcahy2014}.

Under the Affordable Care Act, Federally Qualified Health Center (FQHCs)
eligibility for 340B program was expanded. FQHCs differ from hospitals
as they are explicitly defined as: ``primary care clinics that receive
federal funds to provide healthcare services to underserved communities.
They operate in both rural and urban areas designated as shortage
areas.''\footnote{``Community Health Center Overview''
  \emph{portal.ct.gov/}.
  \url{https://portal.ct.gov/DPH/Family-Health/Community-Health-Centers/Community-Health-Center-Overview\#:~:text=Federally\%20Qualified\%20Health\%20Centers\%20(FQHC)\%20are\%20health\%20centers\%20that\%20receive,designation\%20from\%20the\%20U.S.\%20Dept.}.}
While this policy was originally intended to serve vulnerable
populations, recent work of journalists have shown that 340B
hospital-affiliated clinics were likely to be in more affluent areas
with higher rates of insurance coverage. Previous investigations
conducted by journalists at the New York Times have uncovered many
nonprofit hospitals benefiting from the 340B program such as Bon Secours
nonprofit hospital and beneficiary from the 340b program reaping over
\$100 million dollars in 2021.\footnote{``How a Hospital Chain Used a
  Poor Neighborhood to Turn Huge Profits''
  \emph{https://www.nytimes.com}.
  \url{https://www.nytimes.com/2022/09/24/health/bon-secours-mercy-health-profit-poor-neighborhood.html}.}

As this is a part of a greater independent project, the next section
will focus briefly on explaining the dataframe and the data in more
depth as it is a rather messy dataset that had gone through various
stages of cleaning.

\hypertarget{preliminary-data-discussion}{%
\section{Preliminary Data
Discussion}\label{preliminary-data-discussion}}

The dataframe is composed of the names of FQHCs opened in 2011 and 2019,
with corresponding census data referring to the county in which the FQHC
opened up. To begin analysis, I selected for only a small subset of
variables and identifiers that were necessary for the analysis at hand.
The FQHC data was manually downloaded and cleaned from the Office of
Health Resources and the Census data was retrieved via an API. This is
an ongoing project that aims to integrate more years into the research
and thus generate a more panel-level analysis akin to Scheve and
Schavage. As it currently stands, this paper is largely inspired by the
minimum wage analysis conducted by David Card and Alan Krueger in their
paper: ``Minimum Wages and Employment: A Case Study of the Fast-Food
Industry in New Jersey and Pennsylvania'' in addition to the OSHA case
that looked into various prediction methods.

For the purpose of the OLS/fixed effects, we created new variables such
as ``state\_status'' which indicates if a state is belonging to a
category of states that adopted the ACA vs states that did not adopt it.

\hypertarget{analysis}{%
\section{Analysis}\label{analysis}}

The intuition behind my research proposal can be seen in the set of
visualizations in Figure~\ref{fig-1} as well as the maps in
Figure~\ref{fig-2} and Figure~\ref{fig-3} that depict the openings of
FQHCs between 2011 and 2019.

\newpage{}

From the visualizations, one can notice can notice that the median
family income for the FQHCs in 2019 has grown for both states that
implemented vs have not yet implemented the ACA. However, many events
may have triggered this change including the potential that the states
themselves got richer and therefore it appears that the FQHCs are in
wealthier areas comparatively rather than relatively. What I expected to
see if my hypothesis was correct was not only the FQHCs demographic get
wealthier for ACA implemented states in 2019 to non-ACA implemented
states but that compared to 2014, the difference between the two state
categories is signifcantly higher.

\begin{figure}[tbp]

{\centering \includegraphics{figures/comb_stats.png}

}

\caption{\label{fig-1}FQHC County Median Income compared between 2011
and 2019}

\end{figure}

\begin{figure}[tbp]

{\centering \includegraphics[width=0.5\textwidth,height=\textheight]{figures/map2.png}

}

\caption{\label{fig-2}Map of FQHCs opened in 2011}

\end{figure}

\begin{figure}[tbp]

{\centering \includegraphics[width=0.5\textwidth,height=\textheight]{figures/map1.png}

}

\caption{\label{fig-3}Map of FQHCs opened in 2019}

\end{figure}

To get a better understanding of the demographics, we also created two
summary tables that allowed us to see both regional variation between
states that implemented the ACA vs states that did not implement the
ACA. Nevertheless, the tabular representation of regional variation in
Figure~\ref{fig-4} and Figure~\ref{fig-5} helps us understand further
the discrepancies in demographics such as median family income,
unemployment, rural/urban community.

\newpage{}

\begin{figure}[tbp]

{\centering \includegraphics{figures/sumstat_2.png}

}

\caption{\label{fig-4}Summary Statistics 1}

\end{figure}

\begin{figure}[tbp]

{\centering \includegraphics{figures/sumstat_1.png}

}

\caption{\label{fig-5}Summary Statistics 1}

\end{figure}

From the summary table, we can notice some patterns that is worth
investigating further, for example, with regards to Figure~\ref{fig-4},
the states that never implemented the ACA compared to the states that
eventually did, the median income of the FQHCs is high in all regions
with the exception of the South where it decreased from \$61,112 to
\$59,205.It appears however that states that implemented ACA have a
higher median income (\$69,551) than states that did not (\$67,873.88).
Interestingly however is that unemployment and percentage on cash
assistance is also higher for the counties served by the FQHCs in areas
where the ACA was implemented. Although this is interesting, it is not
enough to deduce a causal effect from. Furthermore, looking at
Figure~\ref{fig-5} we notice that compared to 2011 (when ACA had not
been passed federally), 2019 had a notable increase in median income
across all regions. Prior to the enacting of the ACA, the average median
income is around 59,490 dollars, a figure that grew to 72,419 after the
implementation of ACA. Looking more regionally, we can notice that the
South is notably ranked the lowest in median income compared to other
regions and while the FQHCs in the South grew in income in 2019
(Post-ACA), it is not nearly as high as the other regions. This is also
interesting to note as the many southern states did not implement the
ACA and yet still have a increase in FQHC income (\$50,611 median in
2011 vs \$65,363 median in 2019). Other statistics of note is that
median percentage below poverty grew Post-ACA across all regions from
11.10\% to 26.40\% while median unemployment rate decreased from 9\% to
5.70\% (although this may be due to the aftermaths of the 2008 housing
crisis).

In order to address the variation present in income, we employed a
difference in differences model. While we did also run a fixed effects
model and doing so is feasible (as soon in the equations below), we
opted to focus primarily on the OLS models for ease of interpretability.

The equations we employed for OLS were:

\[I_{it} = \alpha + \beta_{1}(ExpandedStates)_i + \beta_{2}(PostACA)_t + \beta_{3}(ExpandedStates_i x PostACA_t) + \gamma X + \epsilon,\]
where \(I_{it}\) is the log median family income, \(\beta_1\) represents
the coefficient for when a state has expanded and implemented the ACA,
in that case, i = 1 when referring to expanded states and i = 0 when
referring to non expansion states, \(\beta_2\) represents the time
period and in reference to the Federal Passage of the ACA (not to
confused with the state level adoption). In this case, t = 0 to Pre-ACA
and t = 1 to Post-ACA. Finally \(\beta_3\) is the interaction between
the states that have implemented the ACA in the post-ACA time period.
\(\gamma\) represents a vector of coefficients for controls such as
rural/urban and unemployment rate.

A fixed effects equation that would capture the same thing would be:

\[
I_{it} = \alpha_i + \theta_t + \beta_{it}(PostACA) + \gamma_{it}X + \epsilon_{it}
\]

Where \(i\) in this case indicates the state, the \(t\) is referring to
the time. We used \(\alpha\) to represent state fixed effects and
\(\theta\) to represent time fixed effects. Ultimately however due to
the 2 by 2 nature of the data, we opted primarily to use the OLS form in
running our equations.

To run a difference in differences model we have to make some
significant assumptions of our findings which unfortunately based on the
available data set are difficult to confirm. Firstly, we have to make
the parallel trends assumption whereby we suppose that absent treatment
(ACA), the states that would've implemented the ACA would have trended
the same way (in parallel) to the states that did not implement ACA. As
our data is not yet part of a larger panel series, we do not have a
method to confirm this assumption (although it should be noted that it
is impossible to fully confirm the parallel trends assumption).
Potential methods to do so however as the project possesses, would
require utilizing a lags and leads, looking to see if leads,
specifically indicate certain trends that would be happening absent of
treatment.

\hypertarget{results}{%
\section{Results}\label{results}}

Below we can notice the results of the regression.

\begin{figure}[tbp]

{\centering \includegraphics{figures/modelsum.png}

}

\caption{\label{fig-6}Table2: OLS Model Summary}

\end{figure}

The intercept of 10.849 indicates that on average States that would not
have implemented the ACA prior to 2014, would have on average a median
log family income of 10.849. The difference across time between states
that have not implemented the ACA is captured by the coefficient
reflecting the 2014 ACA federal adoption which was .243. This indicates
that from 2011 and 2019, States that were not going to implement the ACA
saw a 24.3\% increase in income in FQHC counties. The difference in
differences was captured by the Effect of Post-ACA Federal Passage on
States where ACA was Implemented which was -.0848 which indicates that
the post the passage of the ACA, the states that implemented ACA
actually had a decrease in FQHC county income of around 8.48\%.

This does not support our hypothesis as we can notice that ACA
implementation actually leads to a decrease in log median family income
in FQHC counties. When pursued further controls, this pattern is still
prevelant as this coefficient remains negative until the long regression
where it is a small positive (however this is not signficant). In the
long regression, we controlled for unemployment rate and Rural/Urban
Designation and found that regardless of ACA impacts, the counties where
the FQHCs are located are most impacted by unemployment or being
rural/urban as a percentage point increase in unemployment rate, ceteris
paribus, is associated with a -5.8\% decrease in income and a rural
county has an median income that is -3.7\% less than their urban
counterparts. Adjusting for these controls allowed us to recognize that
the causal relationship we are supposing between ACA implementation on
county income is not as strong under this experimental design as once
accounting for variables such as unemployment rate and RUCA designation,
the significance diminishes. In employing such controls we are able to
understand some relationships that may be of note.

To test our model's accuracy we also split our data into training and
test data, beginning first by separating out the years of the data and
recombining as there are more FQHCs in 2019 than in 2011 and we didn't
want to over represent one group than the other. The results of the
split dataset can be seen in Figure~\ref{fig-7} with the RMSE being the
lowest for the long regression model that utilized all controls. Both
the training and test RMSE were the lowest out of the dataset. This
makes sense as we only have a small set of variables therefore we avoid
the issue of overfitting while still making sure our prediction is as
accurate as possible in predicting log income. If we want to choose a
model that has a more interpretable coefficient \(\beta_{3}\) we can opt
for the model with unemployment controls as that had the second lowest
training and testing errors. In this version the coefficient is still
negative but is -.035 which translates to the ACA implementation leading
to a -3.5\% change in the log income.

\begin{figure}[tbp]

{\centering \includegraphics{figures/rmse.png}

}

\caption{\label{fig-7}RMSE Table}

\end{figure}

\hypertarget{evaluation}{%
\section{Evaluation}\label{evaluation}}

The findings of this paper were rather surprising given our hypothesis
as well as previous work in the field analyzing the 340B program. From
our initial visualizations it was clear that the number of FQHCs had
increased substantively from 2011 to 2019. This increase also appeared
to correlate with an increase in the income of the areas where the FQHCs
were opening in. When looking at the summary statistics, we were able to
notice that between 2011 and 2019, the counties where FQHCs were opening
were noticeably more wealthy. We wanted to learn if this increase in
wealth was due to the ACA decreasing limitation of qualifying FQHCs and
therefore FQHCs could open in wealthier areas. We also had to be
cognizant that the affects we were associating with the implementation
of the ACA was not due to the fact that maybe the counties were already
on their way to becoming wealthier regardless of ACA implementation
(more economic growth).

To do this, we implemented a difference in differences model with
controls that took into account if an area is urban/rural and their
unemployment rate. It should be noted that there was some colinearity
between urban/rural and unemployment that was not assumed to be
significant enough to skew the analysis. When looking at the average log
median income we noticed that in 2019 for both states that implemented
the ACA and did not implement the ACA the log median income went up. Our
difference in difference simple regression model indicates that
implementation of ACA leads actually to a decrease of 8.5\% in median
income of the FQHCs. While the additional of controls such as
unemployment and rural/urban makes this decrease smaller and more
positive in the case of the long regression, it can be deduced that the
ACA instead of leading to higher income environments of FQHCs, there is
a subtle decrease, indicating that the FQHCs may actually be helping
more vulnerable and low-income areas than previously thought.

This being said, our experimental design has several drawbacks that must
be addressed to must adequately prepare for future experiments and
derive more robust conclusions. The primary issue with a simple before
and after experiment that does not have panel level data is that it
neglects effects of time. The composition of US counties have been
getting wealthier over time and therefore between 2011 and 2019, the
counties of interest grew in wealth mirroring a national pattern.
Therefore, in future studies and in fact where this study is expanding
towards would take into account panel data and run an interrupted
time-series analysis. In doing so, we can account for general increases
in county wealth and more precisely accredit any variation to the ACA.
Similarly, with more panel level data, our fixed effects would also take
into account state level variation and be better able to account if
variation was due simply to states.

\hypertarget{conclusion}{%
\section{Conclusion}\label{conclusion}}

Our original hypothesis purposed that ACA expansion which increased FQHC
eligibility and thus lead to more FQHCs in wealthier eras. However our
difference in differences paints a different story indicating that the
ACA may be helping the FQHCs locate in less wealthy areas compared to
the states that did not implement the ACA.

This provides support for ACA-based regulation of FQHC placement as
perhaps expanding eligibility will not lead to excessive profit seeking
behaviors but rather more FQHCs opening in areas where it would
otherwise not be served. Moreover, this data-set uses county-level data
which may not be accurately reflecting the income demographics of the
FQHC locations as many health centers may be in a ``less wealthy''
county but are located in wealthy zipcodes or in areas where they are
able to still engage in profit-seeking behaviors. In fact according to
the Department of Commerce, geographic income inequality has increased
more than 40\% between 1980 and 2021.\footnote{``Local incomes have
  become more unequal over time'' \emph{https://www.commerce.gov/}.
  \url{https://www.commerce.gov/news/blog/2023/06/geographic-inequality-rise-us\#:~:text=Geographic\%20income\%20inequality\%20has\%20risen,of\%20metropolitan\%20and\%20micropolitan\%20areas.}.}
As the ACA and other pieces of legislation are passed and FQHCs continue
to pop up in number, it is nevertheless vital to continually monitor
where these FQHCs are located in so that they can continue to serve the
population they were designed to help.

\hypertarget{refs}{}

\begin{CSLReferences}{0}{0}\end{CSLReferences}


% END BODY %%%%%%%%%%%%%%%%%%%%%%%%%%%%%%%%%%%%%%%%%%%%%%%%%%%%%%%%%%%%%%%%%%%%%



\end{document}
